% Created 2017-07-26 Wed 03:06
\documentclass[11pt]{article}
\usepackage{verbatim}
\usepackage{luatexja-fontspec}
\usepackage{graphicx}
\usepackage{longtable}
\usepackage{float}
\usepackage{wrapfig}
\usepackage{rotating}
\usepackage[normalem]{ulem}
\usepackage{amsmath}
\usepackage{textcomp}
\usepackage{marvosym}
\usepackage{wasysym}
\usepackage{amssymb}
\usepackage{hyperref}
\tolerance=1000
\setmainjfont[BoldFont=Pingfang SC]{FandolSong}
\setsansjfont{Pingfang SC}
\author{Nasy}
\date{Jul 26, 2017}
\title{Incremental Update}
\hypersetup{
  pdfkeywords={},
  pdfsubject={},
  pdfcreator={Emacs 25.2.1 (Org mode 8.2.10)}}
\begin{document}

\maketitle

\section*{Aim}
\label{sec-1}
Compare the files in the two folders, the B folder than the A folder more files, put these files into a separate folder C. If files already in C folder, pass it.

\section*{Why this}
\label{sec-2}
A friend of mine wants to do this.

\section*{Algorithm}
\label{sec-3}
I think there is no algorithm here, only the nature and operation of the collection.

\section*{Code}
\label{sec-4}

\begin{verbatim}
import os
import shutil


def incremental_update(a_folder_path: str,
                       b_folder_path: str,
                       c_folder_path: str) -> None:
    """Incremental update.

    Args:
        a_folder_path str: A folder path.
        b_folder_path str: B folder path.
        c_folder_path str: C folder path.

    Return:
        None
    """
    # Compare files between 3 folders.
    c_files = (set(os.listdir(a_folder_path)) - set(os.listdir(b_folder_path))
               - set(os.listdir(c_folder_path)))

    if not os.path.isdir(c_folder_path):
        os.mkdir(c_folder_path)

    def _copy(f_path: str) -> None:
        """Copy file from B to C.

        Args:
            f_path: str. The file in B folder.
        """
        shutil.copy(
            os.path.join(b_folder_path, f_path),
            os.path.join(c_folder_path, f_path))

    for f_path in c_files:
        _copy(f_path)
\end{verbatim}
% Emacs 25.2.1 (Org mode 8.2.10)
\end{document}
